%%「論文」,「レター」,「レター(C分冊)」,「技術研究報告」などのテンプレート
%% 1. 「論文」
%% v3.0 [2015/11/14]

%% 4. 「技術研究報告」
\documentclass[technicalreport]{ieicej}
%\usepackage[dvips]{graphicx}
%\usepackage[dvipdfmx]{graphicx,xcolor}
\usepackage[T1]{fontenc}
\usepackage{lmodern}
\usepackage{textcomp}
\usepackage{latexsym}
%\usepackage[fleqn]{amsmath}
%\usepackage{amssymb}

\setcounter{page}{1}

\jtitle{複数人で使用可能な3Dアイデアノートシステムの提案と実装}
\jsubtitle{}
\etitle{}
\esubtitle{}
\authorlist{%
 \authorentry{猪膝 孝之}{Takayuki INOHIZA}{uec}\MembershipNumber{uec}
 \authorentry{田野 俊一}{Shun'ichi TANO}{uec}\MembershipNumber{uec}
 \authorentry{橋山 智訓}{Tomonori HASHIYAMA}{uec}\MembershipNumber{uec}
 \authorentry{丸谷 大樹}{Taiki MARUYA}{uec}\MembershipNumber{uec}
 \authorentry{市野 順子}{Junko ICHINO}{tcu}\MembershipNumber{tcu}
 \authorentry{森 真吾}{Shingo MORI}{tis}\MembershipNumber{tis}
 \authorentry{井出 将弘}{Masahiro IDE}{tis}\MembershipNumber{tis}
% \authorentry[メールアドレス]{和文著者名}{英文著者名}{所属ラベル}
}
\affiliate[uec]{電気通信大学大学院 情報理工学研究科}{Graduate School of Informatics and Engineering, The University of Electro-Communications}
\affiliate[tcu]{東京都市大学 メディア情報学部}{Faculty of Informatics, Tokyo City University}
\affiliate[tis]{TIS株式会社}{TIS Inc.}
%\affiliate[所属ラベル]{和文勤務先\\ 連絡先住所}{英文勤務先\\ 英文連絡先住所}

\begin{document}
\begin{jabstract}
%和文あらまし
近年、HMD(Head Mounted Display) が普及してきた。現在、HMDに関する研究は一人で使用することを想定していたり、特定の場所で使用することを想定していたり、手やペンで描くのみで入力手法が限定されているものが多い。これらの問題点を踏まえた上で本研究では、複数人で使用可能で、どこでも場所を選ばず利用が可能で、直感的で様々な入力が可能なHMDを使用したシステムの提案を行う。これらのコンセプトに基づいたシステムを設計し、実際にプロトタイプシステムの実装を行った。また、評価実験ではシナリオ実験と課題解決実験の二つを行い、シナリオ実験での手順の遂行状況においてはほとんどの手順において成功率100\%を収めた。課題解決実験においても多くの被験者がタスクを実行できることが確認できた。
\end{jabstract}
\begin{jkeyword}
%和文キーワード
3D, メモ書き, 共有, マルチモーダル
\end{jkeyword}
%\begin{eabstract}
%英文アブストラクト
%\end{eabstract}
%\begin{ekeyword}
%英文キーワード
%\end{ekeyword}
\maketitle

\section{はじめに}
アイデアはふとしたときに思い浮かぶことがあり、私達はそれを紙に書き留めたり、PCやスマートフォン等を利用してメモを取ることがある。しかし、アイデアが思い浮かぶのは座って作業しているときだけでなく、外で歩いているときや、机やホワイトボード等がないような場面でも突然思い浮かぶことがある。紙とペンを持ち歩いて思い浮かんだらすぐにメモを取る習慣ができてる人はいいが、そうでない人はアイデアが思い浮かんでも「後でメモをすればいいや」と思ってすぐにメモを取ることを諦めてしまうだろう。実際に日頃からアイデアをメモに記録している人は少なくなってきている傾向がある。思い浮かんだアイデアはできるだけ早くメモを取ったほうが良い。また、アイデアは一人で考えて生み出すものとは限らない。友人との話し合いをしているうちに一人では思いつかなかったアイデアが生まれたり、話が弾んで連鎖的にアイデアが生まれることもある。アイデアを効率的に生み出すための発想法がすでに何種類もあるが、これらの中には複数人で集まって話し合ってアイデアを出すものも多い。

近年、HMD(Head Mounted Display)が普及してきた。HMDに関する研究はすでに多く行われている。医療技術に関する研究や、デザイナー向けの3Dスケッチに関する研究、他には3D空間上に文字や図形を描いたりする研究等もある。近年ではMRも流行化してきており、医療診断・手術計画や屋内外の誘導・案内、埋蔵物確認、都市計画・建築分野でのシミュレーション、アミューズメント産業等の応用分野から大きな期待がされている。

現在、HMDに関する多くの研究は一人で使うことが想定されている。また、特定の場所において使用することを想定している場合も多い。そして、手やペンで描くのみで入力手法が限定されている場合も多い。現状では複数人で使用することを想定していたり、外などの広い空間で利用することを想定していたり、手で描くだけでなく音声でも入力可能なシステムを想定している研究やアプリケーションは少ない。そこで、複数人で使用できて、どこでも利用できて、様々な入力ができるシステムが必要だと考え、本研究の立案に至った。


%\bibliographystyle{sieicej}
%\bibliography{myrefs}
\begin{thebibliography}{99}% 文献数が10未満の時 {9}
\bibitem{}
\end{thebibliography}

\end{document}
